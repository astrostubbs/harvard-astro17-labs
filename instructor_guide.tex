\documentclass[11pt]{article}
\usepackage[margin=1in]{geometry}
\usepackage{hyperref}
\usepackage{booktabs}
\usepackage{longtable}
\usepackage{enumitem}

\title{Astro 17: Extragalactic Astronomy and Cosmology\\
\large Instructor's Guide to Lab Notebooks}
\author{Harvard University, Fall 2025}
\date{}

\begin{document}
\maketitle

\tableofcontents
\newpage

\section{Introduction}

This guide accompanies the Google Colab notebooks for Astro 17, Harvard's introductory course in extragalactic astronomy and cosmology. The notebooks provide hands-on computational exercises covering topics from stellar populations to cosmological distance measurements.

All notebooks are designed to run on Google Colab, a cloud-based Python notebook environment that requires no local software installation.

\section{Google Colab: Benefits and Considerations}

\subsection{Benefits of Using Colab}

\begin{itemize}[leftmargin=*]
    \item \textbf{No local setup required} -- Students can begin working immediately without installing Python, Jupyter, or any packages on their personal computers.
    \item \textbf{Platform independence} -- Works on any computer with a web browser (Windows, Mac, Linux, Chromebook).
    \item \textbf{Pre-installed scientific libraries} -- NumPy, Matplotlib, SciPy, Astropy, and Pandas are available by default.
    \item \textbf{GPU/TPU access} -- For advanced exercises, Colab provides free access to accelerated computing resources.
    \item \textbf{Easy sharing} -- Notebooks can be shared via Google Drive links for collaboration or grading.
    \item \textbf{Built-in AI assistant} -- The Gemini AI integration helps students understand code and debug issues.
\end{itemize}

\subsection{Important Considerations}

\begin{itemize}[leftmargin=*]
    \item \textbf{Google Drive mounting required} -- To access data files, students must mount their Google Drive and navigate to the folder containing the data files. This requires a Google account.

    \item \textbf{Data files must be uploaded} -- Students need to copy the data files from the course repository to their own Google Drive's \texttt{Colab Notebooks} folder (or another folder of their choice, adjusting the path in the notebook accordingly).

    \item \textbf{pip installations do not persist} -- Any packages installed with \texttt{pip install} during a session are lost when the runtime disconnects. Students must re-run pip install cells each time they open a notebook. For example, the \texttt{LineProfileGalaxies} notebook requires \texttt{pip install ipympl} at the start of each session.

    \item \textbf{Runtime timeouts} -- Colab disconnects after periods of inactivity (typically 90 minutes). Students should save work frequently.

    \item \textbf{Internet required} -- Colab is entirely cloud-based and requires an internet connection.
\end{itemize}

\subsection{Student Setup Instructions}

Students should follow these steps before their first lab session:

\begin{enumerate}
    \item Create a Google account if they don't have one.
    \item Go to \url{https://colab.research.google.com}.
    \item In Google Drive, create a folder called \texttt{Colab Notebooks} (or use an existing folder).
    \item Download all data files from the course repository's \texttt{data/} folder.
    \item Upload the data files to their \texttt{Colab Notebooks} folder on Google Drive.
    \item Open each notebook from the course repository in Colab.
    \item When prompted, authorize Colab to access Google Drive.
\end{enumerate}

\section{Notebook Descriptions}

\subsection{Gaussians}
\begin{description}[leftmargin=!,labelwidth=\widthof{\bfseries Required Data:}]
    \item[Topic:] Statistical fitting and Gaussian distributions
    \item[Learning Objectives:] Understanding probability distributions, curve fitting, chi-squared analysis, parameter estimation with uncertainties
    \item[Required Data:] None (synthetic data generated in notebook)
    \item[Prerequisites:] Basic Python, NumPy arrays
    \item[Common Issues:] None -- self-contained notebook
\end{description}

\subsection{Linear\_Regression\_Correlation}
\begin{description}[leftmargin=!,labelwidth=\widthof{\bfseries Required Data:}]
    \item[Topic:] Linear regression and parameter correlation
    \item[Learning Objectives:] Understanding how slope and intercept become correlated when fitting data far from origin; mean subtraction to improve numerical stability
    \item[Required Data:] None (synthetic data generated in notebook)
    \item[Prerequisites:] Basic statistics, curve fitting concepts
    \item[Common Issues:] None -- self-contained notebook
\end{description}

\subsection{Distances}
\begin{description}[leftmargin=!,labelwidth=\widthof{\bfseries Required Data:}]
    \item[Topic:] Cosmological distance measures
    \item[Learning Objectives:] Understanding luminosity distance, angular diameter distance, comoving distance; cosmological redshift; distance modulus
    \item[Required Data:] None (calculations only)
    \item[Prerequisites:] Basic cosmology concepts
    \item[Common Issues:] None -- self-contained notebook
\end{description}

\subsection{gaia}
\begin{description}[leftmargin=!,labelwidth=\widthof{\bfseries Required Data:}]
    \item[Topic:] Hertzsprung-Russell diagram using Gaia data
    \item[Learning Objectives:] Stellar parallax and distance; absolute vs. apparent magnitude; color-magnitude diagrams; main sequence, giant branch, white dwarfs
    \item[Required Data:] \texttt{gaia\_200pc\_sample.fits}
    \item[Prerequisites:] Magnitude system, stellar classification basics
    \item[Common Issues:] Students must correctly calculate absolute magnitude from parallax
\end{description}

\subsection{Galaxy\_Rotation\_Curve\_Fitting}
\begin{description}[leftmargin=!,labelwidth=\widthof{\bfseries Required Data:}]
    \item[Topic:] Galaxy rotation curves and dark matter
    \item[Learning Objectives:] Rotation curve analysis; evidence for dark matter; mass modeling (disk, bulge, halo components)
    \item[Required Data:] None (NGC 7331 data embedded in notebook)
    \item[Prerequisites:] Keplerian dynamics, mass estimation
    \item[Common Issues:] Understanding the disk/halo decomposition
\end{description}

\subsection{LineProfileGalaxies}
\begin{description}[leftmargin=!,labelwidth=\widthof{\bfseries Required Data:}]
    \item[Topic:] Galaxy morphology through brightness profiles
    \item[Learning Objectives:] FITS image analysis; surface brightness profiles; exponential vs. de Vaucouleurs profiles; morphological classification
    \item[Required Data:] Multiple FITS images: \texttt{M74DSS2red.fits}, \texttt{M87SDSSr.fits}, \texttt{M101DSS2red.fits}, \texttt{M84SDSSr.fits}, \texttt{M89SDSSr.fits}, \texttt{M105SDSSr.fits}, \texttt{NGC4874DSS2red.fits}, and others
    \item[Prerequisites:] Galaxy classification, magnitude systems
    \item[Special Setup:] Requires \texttt{pip install ipympl} at start of each session, followed by kernel restart
    \item[Common Issues:] Interactive widget may not work if ipympl not installed; students must restart runtime after pip install
\end{description}

\subsection{NumberCounts}
\begin{description}[leftmargin=!,labelwidth=\widthof{\bfseries Required Data:}]
    \item[Topic:] Galaxy number counts and star-galaxy separation
    \item[Learning Objectives:] Photometric catalogs; PSF vs. Petrosian magnitudes; star-galaxy separation; number counts as cosmological probe
    \item[Required Data:] \texttt{A1942nohead.tsv}
    \item[Prerequisites:] Magnitude systems, basic statistics
    \item[Common Issues:] Understanding the magnitude uncertainty as inverse SNR
\end{description}

\subsection{ClusterLRG\_Rubin}
\begin{description}[leftmargin=!,labelwidth=\widthof{\bfseries Required Data:}]
    \item[Topic:] Galaxy cluster analysis with Rubin Observatory data
    \item[Learning Objectives:] cModel photometry; color-magnitude diagrams; red sequence selection; photometric redshifts
    \item[Required Data:] \texttt{abell360\_rubin\_dp1\_303\_1\_galaxy\_photometry\_v6.fits}
    \item[Prerequisites:] NumberCounts notebook concepts
    \item[Common Issues:] Large file size may slow upload
\end{description}

\subsection{Zshell}
\begin{description}[leftmargin=!,labelwidth=\widthof{\bfseries Required Data:}]
    \item[Topic:] Galaxy luminosity functions at fixed redshift
    \item[Learning Objectives:] Redshift shells; K-corrections; luminosity functions; Schechter function fitting
    \item[Required Data:] \texttt{sdss\_photometry.csv}
    \item[Prerequisites:] Redshift, magnitude systems
    \item[Common Issues:] Understanding the narrow redshift slice selection
\end{description}

\subsection{SN1a\_Hubble\_Fit}
\begin{description}[leftmargin=!,labelwidth=\widthof{\bfseries Required Data:}]
    \item[Topic:] Type Ia supernovae and the Hubble diagram
    \item[Learning Objectives:] Standard candles; distance modulus; Hubble diagram; cosmological parameter estimation
    \item[Required Data:] None (supernova data embedded in notebook)
    \item[Prerequisites:] Distances notebook concepts
    \item[Common Issues:] Understanding distance modulus vs. redshift relationship
\end{description}

\subsection{Diffraction\_Grating\_Lab}
\begin{description}[leftmargin=!,labelwidth=\widthof{\bfseries Required Data:}]
    \item[Topic:] Spectroscopy with diffraction gratings
    \item[Learning Objectives:] Diffraction grating equation; wavelength calibration; spectral line identification
    \item[Required Data:] Student-provided spectrum images (taken in lab)
    \item[Prerequisites:] Basic optics, wavelength concepts
    \item[Common Issues:] Image quality affects measurements; students upload their own images via \texttt{files.upload()}
\end{description}

\section{Required Python Packages}

All notebooks use packages that are pre-installed on Google Colab:

\begin{itemize}
    \item \texttt{numpy} -- Numerical computing
    \item \texttt{matplotlib} -- Plotting
    \item \texttt{scipy} -- Scientific computing (curve fitting, optimization)
    \item \texttt{astropy} -- Astronomical utilities (FITS, units, coordinates)
    \item \texttt{pandas} -- Data manipulation
\end{itemize}

One notebook requires an additional package:
\begin{itemize}
    \item \texttt{ipympl} -- Required by \texttt{LineProfileGalaxies.ipynb} for interactive plotting. Must be installed with \texttt{pip install ipympl} at the start of each session.
\end{itemize}

\section{Data Files Summary}

\begin{longtable}{lll}
\toprule
\textbf{File} & \textbf{Size} & \textbf{Used By} \\
\midrule
\endhead
\texttt{A1942nohead.tsv} & 2.9 MB & NumberCounts \\
\texttt{abell360\_rubin\_dp1\_...fits} & 1.7 MB & ClusterLRG\_Rubin \\
\texttt{gaia\_200pc\_sample.fits} & 360 KB & gaia \\
\texttt{sdss\_photometry.csv} & 220 KB & Zshell \\
\texttt{M74DSS2red.fits} & 4.2 MB & LineProfileGalaxies \\
\texttt{M87SDSSr.fits} & 4.2 MB & LineProfileGalaxies \\
\texttt{M87SDSSu.fits} & 4.2 MB & LineProfileGalaxies \\
\texttt{M101DSS2red.fits} & 4.2 MB & LineProfileGalaxies \\
\texttt{M101WISE3.5.fits} & 4.2 MB & LineProfileGalaxies \\
\texttt{M84SDSSr.fits} & 4.2 MB & LineProfileGalaxies \\
\texttt{M89SDSSr.fits} & 4.2 MB & LineProfileGalaxies \\
\texttt{M105SDSSr.fits} & 4.2 MB & LineProfileGalaxies \\
\texttt{NGC4874DSS2red.fits} & 4.2 MB & LineProfileGalaxies \\
\texttt{DECAMr.fits} & 33.5 MB & LineProfileGalaxies \\
\bottomrule
\end{longtable}

Total data size: approximately 75 MB.

\section{Suggested Notebook Sequence}

A recommended order for working through the notebooks:

\begin{enumerate}
    \item \textbf{Gaussians} -- Foundational statistics
    \item \textbf{Linear\_Regression\_Correlation} -- Fitting techniques
    \item \textbf{gaia} -- Stellar HR diagram
    \item \textbf{Distances} -- Cosmological distances
    \item \textbf{Galaxy\_Rotation\_Curve\_Fitting} -- Dark matter evidence
    \item \textbf{LineProfileGalaxies} -- Galaxy morphology
    \item \textbf{NumberCounts} -- Galaxy catalogs
    \item \textbf{ClusterLRG\_Rubin} -- Galaxy clusters
    \item \textbf{Zshell} -- Luminosity functions
    \item \textbf{SN1a\_Hubble\_Fit} -- Cosmological measurements
    \item \textbf{Diffraction\_Grating\_Lab} -- Spectroscopy (lab-based)
\end{enumerate}

\section{Technical Contact}

For technical questions about these notebooks, contact:

\bigskip
\textbf{Christopher Stubbs}\\
Harvard University\\
\texttt{[email address]}

\section{Acknowledgements}

[Placeholder for acknowledgements of contributors, teaching fellows, and others who helped develop these materials.]

\bigskip

These notebooks were developed with assistance from AI tools including Google Gemini and Claude Code.

\end{document}
